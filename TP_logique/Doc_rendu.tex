\documentclass{article}
\usepackage[utf8]{inputenc}
\usepackage[T1]{fontenc}
\usepackage{xcolor}
\usepackage{listings}
\lstset{%
  %language=Smt2,
  basicstyle   = \ttfamily,
  keywordstyle =    \color{magenta},
  keywordstyle = [2]\color{orange},
  commentstyle =    \color{gray}\itshape,
  stringstyle  =    \color{cyan},
  numbers      = left,
  frame        = single,
  framesep     = 2pt,
  aboveskip    = 1ex
}

% Figures
\usepackage{graphicx}
\graphicspath{{./images/}}

%% Math
\usepackage{amsmath, amssymb}
\newtheorem{defi}{Définition}

% Extra  
\usepackage[left=3cm,right=3cm,top=2cm,bottom=2cm]{geometry}

\begin{document}

\title{Devoir maison: Gestionnaire de packages et logique propositionnelle
  \author{\begin{tabular}{p{13.7cm}}
      Bardagot Lucas\\
      Tuilier Thomas
    \end{tabular}\\
    \hline}
}

\maketitle

\section*{Exercice 1}
Le but est de représenter les conflits et les dépendances entre les différents packages (schématisés dans la figure
ci dessus). Pour réaliser cela il faut dans un premier temps déclarer tous nos packages (chaque package représente
une variable booléenne).
\begin{lstlisting}
  (declare-const a Bool)
  (declare-const b Bool)
  (declare-const c Bool)
  (declare-const d Bool)
  (declare-const e Bool)
  (declare-const f Bool)
  (declare-const g Bool)
  (declare-const h Bool)
  (declare-const i Bool)
  (declare-const j Bool)
\end{lstlisting}

Pour simplifier l'écriture, nous allons utiliser des fonctions. Chaque fonction repésente un package ainsi que ses
dépendances et conflits. \\On a :\\\par A qui est associée au package a
\par B qui est associée au package b
\par C qui est associée au package c
\par ...

\\Dans notre fichier smt on peut écrire ces différentes fonctions de la manière suivante :

\begin{lstlisting}
  (define-fun A() Bool
  (and (or (not a) b) (or (not a) c d) (or (not a) d e) (or
  (not a) d f))
  )

  (define-fun C() Bool
  (and (or (not c) g h i) (or (not c) (not e)))
  )

  (define-fun D() Bool
  (or (not d) h i)
  )

  (define-fun E() Bool
  (and (or (not e) j) (or (not e) (not c)) (or (not e) (not
  i)))
  )

  (define-fun F() Bool
  (or (not f) j)
  )

  (define-fun G() Bool
  (or (not g) (not h)
  )

\end{lstlisting}

On remarque que les packages h, i et j n'ont pas de fonction associée. i et j n'ayant aucune dépendance ou
contrainte , il n'est pas nécéssaire de créer des fonctions. La variable h a un conflit avec le package g,
cependant comme la fonction g représente déjà ce conflit, il n'est pas nécessaire de redéfinir ce conflit
une deuxième fois.\\
\\
Pour finir, maintenant que toutes les fonctions sont définies, il suffit de dire que l'on souhaite que toutes
les fonctions soient respectées.\\
On peut écrire cela de la manière suivante :
\begin{lstlisting}

  (assert (and A B C D E F G))
  (check-sat)
  (get-model)

\end{lstlisting}
Pour chaque valuation, lors que la valeur booléenne d'un package est 1 (est 'true') alors le package est peut être
installé, si la valeur est 0 (est 'false') alors le package ne peut pas être installer.\\Si on veut spécifier
qu'on a déjà des packages actuellement installés, il suffit de modifier la ligne :

\begin{lstlisting}
  (assert (and A B C D E F G))
\end{lstlisting}

En effet, cette ligne représente finalement la formule qui représente nos liaisons.Donc, si on veut qu'un package
p soit pris en compte comme déjà installé, il suffit d'écrire :

\begin{lstlisting}
  (assert (and A B C D E F G p))
\end{lstlisting}

Cela marche aussi avec plusieurs packages. Par exemble si on veut présicer que les packages a, b et g sont déjà installés on peut écrire :
\begin{lstlisting}
  (assert (and A B C D E F G a b g))
\end{lstlisting}

\section*{Exercice 2}
Pour vérifier que le package b ne peut pas être installé si le package h est installé, il suffit d'écrire :
\begin{lstlisting}
  (declare-const b Bool)
  (declare-const h Bool)

  (define-fun H() Bool
  (or (and h (not b)) (and b (not h)))
  )

  (assert H)
  (check-sat)
  (get-model)

\end{lstlisting}

De la même maière que l'exercice 1, on a dans un premier temps déclaré nos variables, puis défini la fonction H et
les commandes. La fonction H repésente la problèmatique de la question 2. En effet, la fonction nous dit que l'on
a deux possibilités:\\
\par Soit h est installé et dans ce cas on ne peut pas avoir b
\par Soit b est installé et dans ce cas on ne peut pas avoir h

\section*{Exercice 3}
\section*{Exercice 4}
\section*{Exercice 5}
\section*{Exercice 6}
\section*{Exercice 7}
\end{document}  
